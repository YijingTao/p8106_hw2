% Options for packages loaded elsewhere
\PassOptionsToPackage{unicode}{hyperref}
\PassOptionsToPackage{hyphens}{url}
%
\documentclass[
]{article}
\usepackage{amsmath,amssymb}
\usepackage{lmodern}
\usepackage{iftex}
\ifPDFTeX
  \usepackage[T1]{fontenc}
  \usepackage[utf8]{inputenc}
  \usepackage{textcomp} % provide euro and other symbols
\else % if luatex or xetex
  \usepackage{unicode-math}
  \defaultfontfeatures{Scale=MatchLowercase}
  \defaultfontfeatures[\rmfamily]{Ligatures=TeX,Scale=1}
\fi
% Use upquote if available, for straight quotes in verbatim environments
\IfFileExists{upquote.sty}{\usepackage{upquote}}{}
\IfFileExists{microtype.sty}{% use microtype if available
  \usepackage[]{microtype}
  \UseMicrotypeSet[protrusion]{basicmath} % disable protrusion for tt fonts
}{}
\makeatletter
\@ifundefined{KOMAClassName}{% if non-KOMA class
  \IfFileExists{parskip.sty}{%
    \usepackage{parskip}
  }{% else
    \setlength{\parindent}{0pt}
    \setlength{\parskip}{6pt plus 2pt minus 1pt}}
}{% if KOMA class
  \KOMAoptions{parskip=half}}
\makeatother
\usepackage{xcolor}
\IfFileExists{xurl.sty}{\usepackage{xurl}}{} % add URL line breaks if available
\IfFileExists{bookmark.sty}{\usepackage{bookmark}}{\usepackage{hyperref}}
\hypersetup{
  pdftitle={p8106 hw2},
  pdfauthor={Yijing Tao yt2785},
  hidelinks,
  pdfcreator={LaTeX via pandoc}}
\urlstyle{same} % disable monospaced font for URLs
\usepackage[margin=1in]{geometry}
\usepackage{color}
\usepackage{fancyvrb}
\newcommand{\VerbBar}{|}
\newcommand{\VERB}{\Verb[commandchars=\\\{\}]}
\DefineVerbatimEnvironment{Highlighting}{Verbatim}{commandchars=\\\{\}}
% Add ',fontsize=\small' for more characters per line
\usepackage{framed}
\definecolor{shadecolor}{RGB}{248,248,248}
\newenvironment{Shaded}{\begin{snugshade}}{\end{snugshade}}
\newcommand{\AlertTok}[1]{\textcolor[rgb]{0.94,0.16,0.16}{#1}}
\newcommand{\AnnotationTok}[1]{\textcolor[rgb]{0.56,0.35,0.01}{\textbf{\textit{#1}}}}
\newcommand{\AttributeTok}[1]{\textcolor[rgb]{0.77,0.63,0.00}{#1}}
\newcommand{\BaseNTok}[1]{\textcolor[rgb]{0.00,0.00,0.81}{#1}}
\newcommand{\BuiltInTok}[1]{#1}
\newcommand{\CharTok}[1]{\textcolor[rgb]{0.31,0.60,0.02}{#1}}
\newcommand{\CommentTok}[1]{\textcolor[rgb]{0.56,0.35,0.01}{\textit{#1}}}
\newcommand{\CommentVarTok}[1]{\textcolor[rgb]{0.56,0.35,0.01}{\textbf{\textit{#1}}}}
\newcommand{\ConstantTok}[1]{\textcolor[rgb]{0.00,0.00,0.00}{#1}}
\newcommand{\ControlFlowTok}[1]{\textcolor[rgb]{0.13,0.29,0.53}{\textbf{#1}}}
\newcommand{\DataTypeTok}[1]{\textcolor[rgb]{0.13,0.29,0.53}{#1}}
\newcommand{\DecValTok}[1]{\textcolor[rgb]{0.00,0.00,0.81}{#1}}
\newcommand{\DocumentationTok}[1]{\textcolor[rgb]{0.56,0.35,0.01}{\textbf{\textit{#1}}}}
\newcommand{\ErrorTok}[1]{\textcolor[rgb]{0.64,0.00,0.00}{\textbf{#1}}}
\newcommand{\ExtensionTok}[1]{#1}
\newcommand{\FloatTok}[1]{\textcolor[rgb]{0.00,0.00,0.81}{#1}}
\newcommand{\FunctionTok}[1]{\textcolor[rgb]{0.00,0.00,0.00}{#1}}
\newcommand{\ImportTok}[1]{#1}
\newcommand{\InformationTok}[1]{\textcolor[rgb]{0.56,0.35,0.01}{\textbf{\textit{#1}}}}
\newcommand{\KeywordTok}[1]{\textcolor[rgb]{0.13,0.29,0.53}{\textbf{#1}}}
\newcommand{\NormalTok}[1]{#1}
\newcommand{\OperatorTok}[1]{\textcolor[rgb]{0.81,0.36,0.00}{\textbf{#1}}}
\newcommand{\OtherTok}[1]{\textcolor[rgb]{0.56,0.35,0.01}{#1}}
\newcommand{\PreprocessorTok}[1]{\textcolor[rgb]{0.56,0.35,0.01}{\textit{#1}}}
\newcommand{\RegionMarkerTok}[1]{#1}
\newcommand{\SpecialCharTok}[1]{\textcolor[rgb]{0.00,0.00,0.00}{#1}}
\newcommand{\SpecialStringTok}[1]{\textcolor[rgb]{0.31,0.60,0.02}{#1}}
\newcommand{\StringTok}[1]{\textcolor[rgb]{0.31,0.60,0.02}{#1}}
\newcommand{\VariableTok}[1]{\textcolor[rgb]{0.00,0.00,0.00}{#1}}
\newcommand{\VerbatimStringTok}[1]{\textcolor[rgb]{0.31,0.60,0.02}{#1}}
\newcommand{\WarningTok}[1]{\textcolor[rgb]{0.56,0.35,0.01}{\textbf{\textit{#1}}}}
\usepackage{graphicx}
\makeatletter
\def\maxwidth{\ifdim\Gin@nat@width>\linewidth\linewidth\else\Gin@nat@width\fi}
\def\maxheight{\ifdim\Gin@nat@height>\textheight\textheight\else\Gin@nat@height\fi}
\makeatother
% Scale images if necessary, so that they will not overflow the page
% margins by default, and it is still possible to overwrite the defaults
% using explicit options in \includegraphics[width, height, ...]{}
\setkeys{Gin}{width=\maxwidth,height=\maxheight,keepaspectratio}
% Set default figure placement to htbp
\makeatletter
\def\fps@figure{htbp}
\makeatother
\setlength{\emergencystretch}{3em} % prevent overfull lines
\providecommand{\tightlist}{%
  \setlength{\itemsep}{0pt}\setlength{\parskip}{0pt}}
\setcounter{secnumdepth}{-\maxdimen} % remove section numbering
\ifLuaTeX
  \usepackage{selnolig}  % disable illegal ligatures
\fi

\title{p8106 hw2}
\author{Yijing Tao yt2785}
\date{2022-03-02}

\begin{document}
\maketitle

\begin{Shaded}
\begin{Highlighting}[]
\FunctionTok{library}\NormalTok{(tidyverse)}
\end{Highlighting}
\end{Shaded}

\begin{verbatim}
## -- Attaching packages --------------------------------------- tidyverse 1.3.1 --
\end{verbatim}

\begin{verbatim}
## v ggplot2 3.3.5     v purrr   0.3.4
## v tibble  3.1.4     v dplyr   1.0.7
## v tidyr   1.1.3     v stringr 1.4.0
## v readr   2.0.1     v forcats 0.5.1
\end{verbatim}

\begin{verbatim}
## -- Conflicts ------------------------------------------ tidyverse_conflicts() --
## x dplyr::filter() masks stats::filter()
## x dplyr::lag()    masks stats::lag()
\end{verbatim}

\begin{Shaded}
\begin{Highlighting}[]
\FunctionTok{library}\NormalTok{(readxl)}
\FunctionTok{library}\NormalTok{(ISLR)}
\FunctionTok{library}\NormalTok{(glmnet)}
\end{Highlighting}
\end{Shaded}

\begin{verbatim}
## 载入需要的程辑包:Matrix
\end{verbatim}

\begin{verbatim}
## 
## 载入程辑包:'Matrix'
\end{verbatim}

\begin{verbatim}
## The following objects are masked from 'package:tidyr':
## 
##     expand, pack, unpack
\end{verbatim}

\begin{verbatim}
## Loaded glmnet 4.1-3
\end{verbatim}

\begin{Shaded}
\begin{Highlighting}[]
\FunctionTok{library}\NormalTok{(caret)}
\end{Highlighting}
\end{Shaded}

\begin{verbatim}
## 载入需要的程辑包:lattice
\end{verbatim}

\begin{verbatim}
## 
## 载入程辑包:'caret'
\end{verbatim}

\begin{verbatim}
## The following object is masked from 'package:purrr':
## 
##     lift
\end{verbatim}

\begin{Shaded}
\begin{Highlighting}[]
\FunctionTok{library}\NormalTok{(corrplot)}
\end{Highlighting}
\end{Shaded}

\begin{verbatim}
## corrplot 0.92 loaded
\end{verbatim}

\begin{Shaded}
\begin{Highlighting}[]
\FunctionTok{library}\NormalTok{(plotmo)}
\end{Highlighting}
\end{Shaded}

\begin{verbatim}
## 载入需要的程辑包:Formula
\end{verbatim}

\begin{verbatim}
## 载入需要的程辑包:plotrix
\end{verbatim}

\begin{verbatim}
## 载入需要的程辑包:TeachingDemos
\end{verbatim}

\begin{Shaded}
\begin{Highlighting}[]
\FunctionTok{library}\NormalTok{(mgcv)}
\end{Highlighting}
\end{Shaded}

\begin{verbatim}
## 载入需要的程辑包:nlme
\end{verbatim}

\begin{verbatim}
## 
## 载入程辑包:'nlme'
\end{verbatim}

\begin{verbatim}
## The following object is masked from 'package:dplyr':
## 
##     collapse
\end{verbatim}

\begin{verbatim}
## This is mgcv 1.8-38. For overview type 'help("mgcv-package")'.
\end{verbatim}

\begin{Shaded}
\begin{Highlighting}[]
\FunctionTok{library}\NormalTok{(earth)}
\end{Highlighting}
\end{Shaded}

\begin{Shaded}
\begin{Highlighting}[]
\NormalTok{college\_df\_nores }\OtherTok{=} \FunctionTok{read\_csv}\NormalTok{(}\StringTok{"./College.csv"}\NormalTok{) }\SpecialCharTok{\%\textgreater{}\%} 
  \FunctionTok{data.frame}\NormalTok{() }\SpecialCharTok{\%\textgreater{}\%} 
  \FunctionTok{na.omit}\NormalTok{() }\SpecialCharTok{\%\textgreater{}\%} 
  \FunctionTok{select}\NormalTok{(}\SpecialCharTok{{-}}\NormalTok{Outstate)}
\end{Highlighting}
\end{Shaded}

\begin{verbatim}
## Rows: 565 Columns: 18
\end{verbatim}

\begin{verbatim}
## -- Column specification --------------------------------------------------------
## Delimiter: ","
## chr  (1): College
## dbl (17): Apps, Accept, Enroll, Top10perc, Top25perc, F.Undergrad, P.Undergr...
\end{verbatim}

\begin{verbatim}
## 
## i Use `spec()` to retrieve the full column specification for this data.
## i Specify the column types or set `show_col_types = FALSE` to quiet this message.
\end{verbatim}

\begin{Shaded}
\begin{Highlighting}[]
\NormalTok{college\_df\_res }\OtherTok{=} \FunctionTok{read\_csv}\NormalTok{(}\StringTok{"./College.csv"}\NormalTok{) }\SpecialCharTok{\%\textgreater{}\%} 
  \FunctionTok{data.frame}\NormalTok{() }\SpecialCharTok{\%\textgreater{}\%} 
  \FunctionTok{na.omit}\NormalTok{() }\SpecialCharTok{\%\textgreater{}\%} 
  \FunctionTok{select}\NormalTok{(Outstate)}
\end{Highlighting}
\end{Shaded}

\begin{verbatim}
## Rows: 565 Columns: 18
\end{verbatim}

\begin{verbatim}
## -- Column specification --------------------------------------------------------
## Delimiter: ","
## chr  (1): College
## dbl (17): Apps, Accept, Enroll, Top10perc, Top25perc, F.Undergrad, P.Undergr...
\end{verbatim}

\begin{verbatim}
## 
## i Use `spec()` to retrieve the full column specification for this data.
## i Specify the column types or set `show_col_types = FALSE` to quiet this message.
\end{verbatim}

\begin{Shaded}
\begin{Highlighting}[]
\NormalTok{college\_df }\OtherTok{=} \FunctionTok{cbind}\NormalTok{(college\_df\_nores, college\_df\_res) }\SpecialCharTok{\%\textgreater{}\%} 
  \FunctionTok{data.frame}\NormalTok{()}

\NormalTok{college\_df2 }\OtherTok{\textless{}{-}} \FunctionTok{model.matrix}\NormalTok{(Outstate }\SpecialCharTok{\textasciitilde{}}\NormalTok{ ., college\_df)[ ,}\SpecialCharTok{{-}}\DecValTok{1}\NormalTok{]}

\FunctionTok{set.seed}\NormalTok{(}\DecValTok{2022}\NormalTok{)}
\NormalTok{trainRows }\OtherTok{\textless{}{-}} \FunctionTok{createDataPartition}\NormalTok{(college\_df}\SpecialCharTok{$}\NormalTok{Outstate, }\AttributeTok{p =}\NormalTok{ .}\DecValTok{8}\NormalTok{, }\AttributeTok{list =}\NormalTok{ F)}

\CommentTok{\# matrix of predictors (glmnet uses input matrix)}
\NormalTok{x1 }\OtherTok{\textless{}{-}}\NormalTok{ college\_df2[trainRows,]}
\CommentTok{\# vector of response}
\NormalTok{y1 }\OtherTok{\textless{}{-}}\NormalTok{ college\_df}\SpecialCharTok{$}\NormalTok{Outstate[trainRows]}
\NormalTok{train }\OtherTok{\textless{}{-}}\NormalTok{ college\_df[trainRows,]}
\CommentTok{\# matrix of predictors (glmnet uses input matrix)}
\NormalTok{x2 }\OtherTok{\textless{}{-}}\NormalTok{ college\_df2[}\SpecialCharTok{{-}}\NormalTok{trainRows,]}
\CommentTok{\# vector of response}
\NormalTok{y2 }\OtherTok{\textless{}{-}}\NormalTok{ college\_df}\SpecialCharTok{$}\NormalTok{Outstate[}\SpecialCharTok{{-}}\NormalTok{trainRows]}
\NormalTok{test }\OtherTok{\textless{}{-}}\NormalTok{ college\_df[}\SpecialCharTok{{-}}\NormalTok{trainRows,]}

\NormalTok{ctrl1 }\OtherTok{\textless{}{-}} \FunctionTok{trainControl}\NormalTok{(}\AttributeTok{method =} \StringTok{"cv"}\NormalTok{, }\AttributeTok{number =} \DecValTok{10}\NormalTok{)}
\end{Highlighting}
\end{Shaded}

\hypertarget{a-perform-exploratory-data-analysis-using-the-training-data-e.g.-scatter-plots-of-response-vs.-predictors.}{%
\subsection{(a) Perform exploratory data analysis using the training
data (e.g., scatter plots of response
vs.~predictors).}\label{a-perform-exploratory-data-analysis-using-the-training-data-e.g.-scatter-plots-of-response-vs.-predictors.}}

\begin{Shaded}
\begin{Highlighting}[]
\NormalTok{theme1 }\OtherTok{\textless{}{-}} \FunctionTok{trellis.par.get}\NormalTok{()}
\NormalTok{theme1}\SpecialCharTok{$}\NormalTok{plot.symbol}\SpecialCharTok{$}\NormalTok{col }\OtherTok{\textless{}{-}} \FunctionTok{rgb}\NormalTok{(.}\DecValTok{2}\NormalTok{, .}\DecValTok{4}\NormalTok{, .}\DecValTok{2}\NormalTok{, .}\DecValTok{5}\NormalTok{)}
\NormalTok{theme1}\SpecialCharTok{$}\NormalTok{plot.symbol}\SpecialCharTok{$}\NormalTok{pch }\OtherTok{\textless{}{-}} \DecValTok{16}
\NormalTok{theme1}\SpecialCharTok{$}\NormalTok{plot.line}\SpecialCharTok{$}\NormalTok{col }\OtherTok{\textless{}{-}} \FunctionTok{rgb}\NormalTok{(.}\DecValTok{8}\NormalTok{, .}\DecValTok{1}\NormalTok{, .}\DecValTok{1}\NormalTok{, }\DecValTok{1}\NormalTok{)}
\NormalTok{theme1}\SpecialCharTok{$}\NormalTok{plot.line}\SpecialCharTok{$}\NormalTok{lwd }\OtherTok{\textless{}{-}} \DecValTok{2}
\NormalTok{theme1}\SpecialCharTok{$}\NormalTok{strip.background}\SpecialCharTok{$}\NormalTok{col }\OtherTok{\textless{}{-}} \FunctionTok{rgb}\NormalTok{(.}\DecValTok{0}\NormalTok{, .}\DecValTok{2}\NormalTok{, .}\DecValTok{6}\NormalTok{, .}\DecValTok{2}\NormalTok{)}
\FunctionTok{trellis.par.set}\NormalTok{(theme1)}

\FunctionTok{featurePlot}\NormalTok{(}\AttributeTok{x =}\NormalTok{ train[,}\DecValTok{2}\SpecialCharTok{:}\DecValTok{17}\NormalTok{], }
            \AttributeTok{y =}\NormalTok{ train[,}\DecValTok{18}\NormalTok{], }
            \AttributeTok{plot =} \StringTok{"scatter"}\NormalTok{, }
            \AttributeTok{span =}\NormalTok{ .}\DecValTok{5}\NormalTok{, }
            \AttributeTok{labels =} \FunctionTok{c}\NormalTok{(}\StringTok{"Predictors"}\NormalTok{,}\StringTok{"Y"}\NormalTok{),}
            \AttributeTok{type =} \FunctionTok{c}\NormalTok{(}\StringTok{"p"}\NormalTok{),}
            \AttributeTok{layout =} \FunctionTok{c}\NormalTok{(}\DecValTok{4}\NormalTok{, }\DecValTok{5}\NormalTok{))}
\end{Highlighting}
\end{Shaded}

\includegraphics{p8106_hw2_files/figure-latex/unnamed-chunk-3-1.pdf}

\hypertarget{b-fit-smoothing-spline-models-using-terminal-as-the-only-predictor-of-outstate-for-a-range-of-degrees-of-freedom-as-well-as-the-degree-of-freedom-obtained-by-generalized-cross-validation-and-plot-the-resulting-fits.-describe-the-results-obtained.}{%
\subsection{(b) Fit smoothing spline models using Terminal as the only
predictor of Outstate for a range of degrees of freedom, as well as the
degree of freedom obtained by generalized cross-validation, and plot the
resulting fits. Describe the results
obtained.}\label{b-fit-smoothing-spline-models-using-terminal-as-the-only-predictor-of-outstate-for-a-range-of-degrees-of-freedom-as-well-as-the-degree-of-freedom-obtained-by-generalized-cross-validation-and-plot-the-resulting-fits.-describe-the-results-obtained.}}

\begin{Shaded}
\begin{Highlighting}[]
\NormalTok{fit.ss }\OtherTok{\textless{}{-}} \FunctionTok{smooth.spline}\NormalTok{(train}\SpecialCharTok{$}\NormalTok{Terminal, train}\SpecialCharTok{$}\NormalTok{Outstate, }\AttributeTok{cv =} \ConstantTok{TRUE}\NormalTok{)}
\end{Highlighting}
\end{Shaded}

\begin{verbatim}
## Warning in smooth.spline(train$Terminal, train$Outstate, cv = TRUE): cross-
## validation with non-unique 'x' values seems doubtful
\end{verbatim}

\begin{Shaded}
\begin{Highlighting}[]
\NormalTok{fit.ss}\SpecialCharTok{$}\NormalTok{df}
\end{Highlighting}
\end{Shaded}

\begin{verbatim}
## [1] 4.458339
\end{verbatim}

\begin{Shaded}
\begin{Highlighting}[]
\NormalTok{Terminal.grid }\OtherTok{\textless{}{-}} \FunctionTok{seq}\NormalTok{(}\AttributeTok{from =} \DecValTok{14}\NormalTok{, }\AttributeTok{to =} \DecValTok{110}\NormalTok{, }\AttributeTok{by =} \DecValTok{1}\NormalTok{)}
\NormalTok{pred.ss }\OtherTok{\textless{}{-}} \FunctionTok{predict}\NormalTok{(fit.ss,}
                   \AttributeTok{x =}\NormalTok{ Terminal.grid)}

\NormalTok{pred.ss.df }\OtherTok{\textless{}{-}} \FunctionTok{data.frame}\NormalTok{(}\AttributeTok{pred =}\NormalTok{ pred.ss}\SpecialCharTok{$}\NormalTok{y,}
                         \AttributeTok{Terminal =}\NormalTok{ Terminal.grid)}

\NormalTok{p }\OtherTok{\textless{}{-}} \FunctionTok{ggplot}\NormalTok{(}\AttributeTok{data =}\NormalTok{ train, }\FunctionTok{aes}\NormalTok{(}\AttributeTok{x =}\NormalTok{ Terminal, }\AttributeTok{y =}\NormalTok{ Outstate)) }\SpecialCharTok{+}
     \FunctionTok{geom\_point}\NormalTok{(}\AttributeTok{color =} \FunctionTok{rgb}\NormalTok{(.}\DecValTok{2}\NormalTok{, .}\DecValTok{4}\NormalTok{, .}\DecValTok{2}\NormalTok{, .}\DecValTok{5}\NormalTok{))}
\NormalTok{p }\SpecialCharTok{+}
\FunctionTok{geom\_line}\NormalTok{(}\FunctionTok{aes}\NormalTok{(}\AttributeTok{x =}\NormalTok{ Terminal, }\AttributeTok{y =}\NormalTok{ pred), }\AttributeTok{data =}\NormalTok{ pred.ss.df,}
          \AttributeTok{color =} \FunctionTok{rgb}\NormalTok{(.}\DecValTok{8}\NormalTok{, .}\DecValTok{1}\NormalTok{, .}\DecValTok{1}\NormalTok{, }\DecValTok{1}\NormalTok{)) }\SpecialCharTok{+} \FunctionTok{theme\_bw}\NormalTok{()}
\end{Highlighting}
\end{Shaded}

\includegraphics{p8106_hw2_files/figure-latex/unnamed-chunk-4-1.pdf}
\textbf{From the plot we can see that the model we fit will lead to a
smooth line, and the trend of the smooth line is the same as the trend
of the points of the real data set. So we can consider the model is a
good fit.}

\hypertarget{c-fit-a-generalized-additive-model-gam-using-all-the-predictors.-plot-the-results-and-explain-your-findings.-report-the-test-error.}{%
\subsection{(c) Fit a generalized additive model (GAM) using all the
predictors. Plot the results and explain your findings. Report the test
error.}\label{c-fit-a-generalized-additive-model-gam-using-all-the-predictors.-plot-the-results-and-explain-your-findings.-report-the-test-error.}}

\begin{Shaded}
\begin{Highlighting}[]
\NormalTok{gam.fit }\OtherTok{\textless{}{-}} \FunctionTok{gam}\NormalTok{(Outstate }\SpecialCharTok{\textasciitilde{}}\NormalTok{ Apps}\SpecialCharTok{+}\NormalTok{Accept}\SpecialCharTok{+}\NormalTok{Enroll}\SpecialCharTok{+}\FunctionTok{s}\NormalTok{(Top10perc)}\SpecialCharTok{+}\NormalTok{Top25perc}\SpecialCharTok{+}\FunctionTok{s}\NormalTok{(F.Undergrad)}\SpecialCharTok{+}\NormalTok{P.Undergrad}\SpecialCharTok{+}\FunctionTok{s}\NormalTok{(Room.Board)}\SpecialCharTok{+}\FunctionTok{s}\NormalTok{(Books)}\SpecialCharTok{+}\NormalTok{Personal}\SpecialCharTok{+}\FunctionTok{s}\NormalTok{(PhD)}\SpecialCharTok{+}\NormalTok{Terminal}\SpecialCharTok{+}\FunctionTok{s}\NormalTok{(S.F.Ratio)}\SpecialCharTok{+}\NormalTok{perc.alumni}\SpecialCharTok{+}\FunctionTok{s}\NormalTok{(Expend)}\SpecialCharTok{+}\FunctionTok{s}\NormalTok{(Grad.Rate), }\AttributeTok{data =}\NormalTok{ train)}

\FunctionTok{plot}\NormalTok{(gam.fit)}
\end{Highlighting}
\end{Shaded}

\includegraphics{p8106_hw2_files/figure-latex/unnamed-chunk-5-1.pdf}
\includegraphics{p8106_hw2_files/figure-latex/unnamed-chunk-5-2.pdf}
\includegraphics{p8106_hw2_files/figure-latex/unnamed-chunk-5-3.pdf}
\includegraphics{p8106_hw2_files/figure-latex/unnamed-chunk-5-4.pdf}
\includegraphics{p8106_hw2_files/figure-latex/unnamed-chunk-5-5.pdf}
\includegraphics{p8106_hw2_files/figure-latex/unnamed-chunk-5-6.pdf}
\includegraphics{p8106_hw2_files/figure-latex/unnamed-chunk-5-7.pdf}
\includegraphics{p8106_hw2_files/figure-latex/unnamed-chunk-5-8.pdf}

\begin{Shaded}
\begin{Highlighting}[]
\NormalTok{gam.pred }\OtherTok{\textless{}{-}} \FunctionTok{predict}\NormalTok{(gam.fit, }\AttributeTok{newdata =}\NormalTok{ test[,}\DecValTok{2}\SpecialCharTok{:}\DecValTok{17}\NormalTok{])}
\CommentTok{\# test error}
\NormalTok{test\_error\_gam }\OtherTok{=} \FunctionTok{mean}\NormalTok{((gam.pred }\SpecialCharTok{{-}}\NormalTok{ y2)}\SpecialCharTok{\^{}}\DecValTok{2}\NormalTok{)}
\NormalTok{test\_error\_gam}
\end{Highlighting}
\end{Shaded}

\begin{verbatim}
## [1] 3529399
\end{verbatim}

\textbf{After plotting the gam model, I found that only the variebles
``Top10perc'', ``F.Undergrad'', ``Room.Board'', ``Books'', ``PhD'',
``S.F.Ratio Expend'' and ``Grad.Rate'' are non-linear, so we only add
``s()'' to these variables when building the gam model. The test error
is \ensuremath{3.5293992\times 10^{6}}.}

\hypertarget{d-train-a-multivariate-adaptive-regression-spline-mars-model-using-all-the-predictors.-report-the-final-model.-present-the-partial-dependence-plot-of-an-arbitrary-predictor-in-your-final-model.-report-the-test-error.}{%
\subsection{(d) Train a multivariate adaptive regression spline (MARS)
model using all the predictors. Report the final model. Present the
partial dependence plot of an arbitrary predictor in your final model.
Report the test
error.}\label{d-train-a-multivariate-adaptive-regression-spline-mars-model-using-all-the-predictors.-report-the-final-model.-present-the-partial-dependence-plot-of-an-arbitrary-predictor-in-your-final-model.-report-the-test-error.}}

\begin{Shaded}
\begin{Highlighting}[]
\NormalTok{mars\_grid }\OtherTok{\textless{}{-}} \FunctionTok{expand.grid}\NormalTok{(}\AttributeTok{degree =} \DecValTok{0}\SpecialCharTok{:}\DecValTok{3}\NormalTok{, }
                         \AttributeTok{nprune =} \DecValTok{2}\SpecialCharTok{:}\DecValTok{15}\NormalTok{)}

\FunctionTok{set.seed}\NormalTok{(}\DecValTok{2}\NormalTok{)}
\NormalTok{mars.fit }\OtherTok{\textless{}{-}} \FunctionTok{train}\NormalTok{(}\AttributeTok{x =}\NormalTok{ train[,}\DecValTok{2}\SpecialCharTok{:}\DecValTok{17}\NormalTok{], }
\NormalTok{                  y1,}
                  \AttributeTok{method =} \StringTok{"earth"}\NormalTok{,}
                  \AttributeTok{tuneGrid =}\NormalTok{ mars\_grid,}
                  \AttributeTok{trControl =}\NormalTok{ ctrl1)}
\end{Highlighting}
\end{Shaded}

\begin{verbatim}
## Warning: model fit failed for Fold01: degree=0, nprune=15 Error in forward.pass(x, y, yw, weights, trace, degree, penalty, nk, thresh,  : 
##   degree 0 is not greater than 0
\end{verbatim}

\begin{verbatim}
## Warning: model fit failed for Fold02: degree=0, nprune=15 Error in forward.pass(x, y, yw, weights, trace, degree, penalty, nk, thresh,  : 
##   degree 0 is not greater than 0
\end{verbatim}

\begin{verbatim}
## Warning: model fit failed for Fold03: degree=0, nprune=15 Error in forward.pass(x, y, yw, weights, trace, degree, penalty, nk, thresh,  : 
##   degree 0 is not greater than 0
\end{verbatim}

\begin{verbatim}
## Warning: model fit failed for Fold04: degree=0, nprune=15 Error in forward.pass(x, y, yw, weights, trace, degree, penalty, nk, thresh,  : 
##   degree 0 is not greater than 0
\end{verbatim}

\begin{verbatim}
## Warning: model fit failed for Fold05: degree=0, nprune=15 Error in forward.pass(x, y, yw, weights, trace, degree, penalty, nk, thresh,  : 
##   degree 0 is not greater than 0
\end{verbatim}

\begin{verbatim}
## Warning: model fit failed for Fold06: degree=0, nprune=15 Error in forward.pass(x, y, yw, weights, trace, degree, penalty, nk, thresh,  : 
##   degree 0 is not greater than 0
\end{verbatim}

\begin{verbatim}
## Warning: model fit failed for Fold07: degree=0, nprune=15 Error in forward.pass(x, y, yw, weights, trace, degree, penalty, nk, thresh,  : 
##   degree 0 is not greater than 0
\end{verbatim}

\begin{verbatim}
## Warning: model fit failed for Fold08: degree=0, nprune=15 Error in forward.pass(x, y, yw, weights, trace, degree, penalty, nk, thresh,  : 
##   degree 0 is not greater than 0
\end{verbatim}

\begin{verbatim}
## Warning: model fit failed for Fold09: degree=0, nprune=15 Error in forward.pass(x, y, yw, weights, trace, degree, penalty, nk, thresh,  : 
##   degree 0 is not greater than 0
\end{verbatim}

\begin{verbatim}
## Warning: model fit failed for Fold10: degree=0, nprune=15 Error in forward.pass(x, y, yw, weights, trace, degree, penalty, nk, thresh,  : 
##   degree 0 is not greater than 0
\end{verbatim}

\begin{verbatim}
## Warning in nominalTrainWorkflow(x = x, y = y, wts = weights, info = trainInfo, :
## There were missing values in resampled performance measures.
\end{verbatim}

\begin{verbatim}
## Warning in train.default(x = train[, 2:17], y1, method = "earth", tuneGrid =
## mars_grid, : missing values found in aggregated results
\end{verbatim}

\begin{Shaded}
\begin{Highlighting}[]
\NormalTok{pdp}\SpecialCharTok{::}\FunctionTok{partial}\NormalTok{(mars.fit, }\AttributeTok{pred.var =} \FunctionTok{c}\NormalTok{(}\StringTok{"Apps"}\NormalTok{), }\AttributeTok{grid.resolution =} \DecValTok{10}\NormalTok{) }\SpecialCharTok{\%\textgreater{}\%}
  \FunctionTok{autoplot}\NormalTok{()}
\end{Highlighting}
\end{Shaded}

\begin{verbatim}
## Warning: Use of `object[[1L]]` is discouraged. Use `.data[[1L]]` instead.
\end{verbatim}

\begin{verbatim}
## Warning: Use of `object[["yhat"]]` is discouraged. Use `.data[["yhat"]]`
## instead.
\end{verbatim}

\includegraphics{p8106_hw2_files/figure-latex/unnamed-chunk-6-1.pdf}

\begin{Shaded}
\begin{Highlighting}[]
\NormalTok{mars.fit}\SpecialCharTok{$}\NormalTok{bestTune}
\end{Highlighting}
\end{Shaded}

\begin{verbatim}
##    nprune degree
## 25     12      1
\end{verbatim}

\begin{Shaded}
\begin{Highlighting}[]
\FunctionTok{coef}\NormalTok{(mars.fit}\SpecialCharTok{$}\NormalTok{finalModel) }
\end{Highlighting}
\end{Shaded}

\begin{verbatim}
##         (Intercept)     h(16262-Expend)  h(5620-Room.Board) h(1365-F.Undergrad) 
##       15180.8099938          -0.5587913          -0.8251141          -1.7285188 
##   h(32-perc.alumni)        h(Apps-1422)       h(Enroll-911)       h(911-Enroll) 
##         -43.3839576           0.4489395          -1.9054530           5.1918949 
##     h(83-Grad.Rate)    h(1323-Personal)           h(PhD-81)      h(1228-Accept) 
##         -23.1058912           0.8852083          54.3923953          -2.3277448
\end{verbatim}

\begin{Shaded}
\begin{Highlighting}[]
\NormalTok{mars.pred }\OtherTok{\textless{}{-}} \FunctionTok{predict}\NormalTok{(mars.fit, }\AttributeTok{newdata =}\NormalTok{ test[,}\DecValTok{2}\SpecialCharTok{:}\DecValTok{17}\NormalTok{])}
\CommentTok{\# test error}
\NormalTok{test\_error\_mars }\OtherTok{=} \FunctionTok{mean}\NormalTok{((mars.pred }\SpecialCharTok{{-}}\NormalTok{ y2)}\SpecialCharTok{\^{}}\DecValTok{2}\NormalTok{)}
\NormalTok{test\_error\_mars}
\end{Highlighting}
\end{Shaded}

\begin{verbatim}
## [1] 3699180
\end{verbatim}

\textbf{I presented the partial dependence plot of ``Apps'' in my final
model. The test error is \ensuremath{3.6991801\times 10^{6}}.}

\hypertarget{e-in-this-data-example-do-you-prefer-the-use-of-mars-model-over-a-linear-model-when-predicting-the-out-of-state-tuition-why}{%
\subsection{(e) In this data example, do you prefer the use of MARS
model over a linear model when predicting the out-of-state tuition?
Why?}\label{e-in-this-data-example-do-you-prefer-the-use-of-mars-model-over-a-linear-model-when-predicting-the-out-of-state-tuition-why}}

\begin{Shaded}
\begin{Highlighting}[]
\FunctionTok{set.seed}\NormalTok{(}\DecValTok{2}\NormalTok{)}
\NormalTok{lm.fit }\OtherTok{\textless{}{-}} \FunctionTok{train}\NormalTok{(}\AttributeTok{x =}\NormalTok{ train[,}\DecValTok{2}\SpecialCharTok{:}\DecValTok{17}\NormalTok{], }
\NormalTok{                  y1,}
                  \AttributeTok{method =} \StringTok{"lm"}\NormalTok{,}
                  \AttributeTok{trControl =}\NormalTok{ ctrl1)}
\end{Highlighting}
\end{Shaded}

\begin{Shaded}
\begin{Highlighting}[]
\NormalTok{resamp }\OtherTok{\textless{}{-}} \FunctionTok{resamples}\NormalTok{(}\FunctionTok{list}\NormalTok{(}\AttributeTok{LN =}\NormalTok{ lm.fit,}
                         \AttributeTok{MARS =}\NormalTok{ mars.fit))}
\FunctionTok{summary}\NormalTok{(resamp)}
\end{Highlighting}
\end{Shaded}

\begin{verbatim}
## 
## Call:
## summary.resamples(object = resamp)
## 
## Models: LN, MARS 
## Number of resamples: 10 
## 
## MAE 
##          Min.  1st Qu.   Median     Mean  3rd Qu.     Max. NA's
## LN   1378.324 1536.575 1585.856 1580.324 1669.019 1693.438    0
## MARS 1072.662 1202.016 1356.984 1313.225 1425.515 1447.072    0
## 
## RMSE 
##          Min.  1st Qu.   Median     Mean  3rd Qu.     Max. NA's
## LN   1825.570 1881.487 1937.965 1975.433 2045.672 2244.553    0
## MARS 1442.274 1532.909 1709.697 1687.899 1804.998 1951.304    0
## 
## Rsquared 
##           Min.   1st Qu.    Median      Mean   3rd Qu.      Max. NA's
## LN   0.6389549 0.6980285 0.7102279 0.7180683 0.7484025 0.7839554    0
## MARS 0.7248618 0.7565213 0.7761406 0.7928721 0.8446916 0.8666082    0
\end{verbatim}

\begin{Shaded}
\begin{Highlighting}[]
\FunctionTok{bwplot}\NormalTok{(resamp, }\AttributeTok{metric =} \StringTok{"RMSE"}\NormalTok{)}
\end{Highlighting}
\end{Shaded}

\includegraphics{p8106_hw2_files/figure-latex/unnamed-chunk-8-1.pdf}
\textbf{I prefer to use MARS model since it has a smaller RMSE}

\end{document}
